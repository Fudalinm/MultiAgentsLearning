\section{Wstęp}

Systemy wieloagentowe to rodzaj komputerowego systemu złożonego z
wielu współdziałających inteligentnych agentów. Mogą rozwiązywać
problemy trudne lub niewykonalne dla indywidualnych agentów\cite{multiagent_system_wiki}.

\subsection{Opis problemu i założeń}
Nasz projekt skupia się na zagadnieniu Ofiary i Łowcy, na
przykład watahy wilków i stada owiec. Zadaniem łowców jest upolowanie
jak największej liczby ofiar w jak najkrótszym czasie, a zadaniem ofiar
jest osiągnięcie jak najdłuższego czasu życia.

Problem zamodelowaliśmy jako dwu-wymiarową siatkę, po której mogą
przemieszczać się agenci (ofiary i łowcy). Każdy z agentów posiada
ten sam zestaw określających go parametrów, które można jednak modyfikować
osobno dla każdego z nich lub dla całej grupy. Agenci mogą przemieszczać się
w czterech podstawowych kierunkach, mają ograniczone pole widzenia (1 pole przed sobą)
oraz określoną prędkość, która wyraża się w ilości pól które mogą pokonać
w jednym kroku symulacji.
