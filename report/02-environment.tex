\section{Opis narzędzi}
Przed rozpoczęciem projektu przeprowadziliśmy badania
istniejących rozwiązań. Interesowały nas głównie projekty
napisane w języku Python, ponieważ nasz zespół czuł się w nim
najpewniej.

Ciekawym rozwiązaniem okazał się być system służący do 
trenowania agentów udostępniony przez grupę OpenAI o nazwie
\textit{Gym}\cite{openai_gym}. Docelowo jednak, udostępniona biblioteka
nie była projektowana z myślą o systemach wieloagentowych. Jej głównym zastosowaniem
są systemy, gdzie każdy agent może pełnić dowolną funkcję, jak na przykład gry dla dwóch
osób o identycznych zasadach dla obu graczy.
Udało nam się jednak znaleźć wersję biblioteki OpenAi rozwiniętą
o wsparcie dla wielu agentów\cite{ma_gym}. W odróżnieniu od oryginalnej biblioteki
każdy agent posiada własne parametry, co pozwala na indywidualną naukę każdego z nich.
Takie rozwiązanie z kolei umożliwia modelowanie bardziej zaawansowanych zachowań i systemów.
